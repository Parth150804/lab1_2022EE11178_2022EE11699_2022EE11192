\documentclass[a4paper,12pt]{article}
\usepackage{graphicx}
\usepackage{hyperref}
\usepackage{amsmath}
\usepackage{listings}
\usepackage{xcolor}
\usepackage{hyperref}
\usepackage{verbatim}
\usepackage{graphicx}
\usepackage{tikz}
\usepackage[utf8]{inputenc}

\usepackage[a4paper,margin=1in]{geometry}
\title{C Lab: Submission Report}

\author{Team Members: \textbf{Parth Gautam, Rohit Rajput}\\ Entry Numbers: \textbf{2022EE11178, 2022EE11699}}

\date{}

\begin{document}
\maketitle

\section{Introduction}

This project involves the development of a command-line spreadsheet program. The spreadsheet operates as a grid of cells, each capable of storing integer values or formulas that reference other cells. Users can interact with the spreadsheet through a command-line interface, allowing them to set values, apply arithmetic operations, and use built-in functions such as \texttt{MIN}, \texttt{MAX}, \texttt{AVG}, \texttt{SUM}, \texttt{STDEV} and \texttt{SLEEP}.  

The program supports dynamic recalculations, ensuring that dependent cells update automatically when their referenced values change. Additionally, it includes error handling for invalid commands, operations on non-existent cells, division by zero, and circular dependencies in formulas.  

To enhance usability, the spreadsheet provides navigation commands for scrolling across large grids and features such as \texttt{disable\_output} and \texttt{enable\_\\output} to control display settings. The implementation prioritizes efficiency, leveraging a Directed Acyclic Graph (DAG) to track dependencies and minimize redundant calculations.  

This report outlines the key design decisions, implementation challenges, error-handling mechanisms, and overall software architecture of the spreadsheet program.  

\section{Design Decisions}
\subsection{Data Structures}
The spreadsheet is implemented using a 2D array of structs, where each cell stores:
\begin{itemize}
    \item A value (iInteger value of the cell)
    \item An expression (Expression (if any) stored as a string)
    \item dependencies list (Array of pointers to nodes on which the cell depends)
    \item dependents list (Array of pointers to nodes which depends on the cell)
    \item dependency count (Number of dependencies)
    \item dependent count (Number of dependents)
    \item an error flag (to check whether the current cell has error or not)
    \item cell expression (after parsing)
\end{itemize}
We use an adjacency list representation for tracking dependencies between cells.

\subsection{Command Parsing}
Commands are parsed using a token-based approach, differentiating between cell assignments, arithmetic operations, function calls, and control inputs.

\subsection{Efficient Recalculation}
To ensure efficient updates, we use a Directed Acyclic Graph (DAG) structure to detect dependencies and avoid redundant computations.

\section{Challenges Faced}
\begin{itemize}
    \item Handling circular dependencies in formulas required implementing cycle detection using DFS.
    \item Ensuring efficient scrolling and display of large sheets within a 10x10 visible grid.
    \item Implementing precise error propagation for operations like division by zero.
\end{itemize}

\section{Error Handling and Edge Cases}
\subsection{Invalid Commands}
Commands not following the expected format result in an \texttt{unrecognized cmd} error.

\subsection{Invalid Cell References}
Accessing non-existent cells (e.g., A1000 in a 2x2 sheet) raises an error.

\subsection{Division by Zero Handling}
If a formula evaluates to division by zero, all dependent cells display \texttt{ERR} and propagate the error.

\title{Command-Line Spreadsheet - Software Architecture}
\author{}
\date{}
\maketitle

\section{Project Structure}
\begin{verbatim}
COP290-2402-C-RPM/
|-- spreadsheet-v1/
|   |-- include/          # Header files
|   |   |-- cl_interface.h
|   |   |-- commands.h
|   |   |-- input_handling.h
|   |-- src/              # Source files
|   |   |-- cl_interface.c
|   |   |-- commands.c
|   |   |-- input_handling.c
|   |   |-- main.c
|   |-- test/             # Testing files
|   |   |-- testing_regex.c
|   |   |-- test_inputs.txt
|   |   |-- test_results.txt
|   |-- Makefile          # Build rules
|   |-- .gitignore        # Ignore files
|   |-- README.md         # Documentation
\end{verbatim}

\section{Data Structures}

\subsection{Cell Structure (\texttt{cell\_t})}
\begin{verbatim}
+--------------------+
| cell_t            |
|--------------------|
| int value         |  <-- Stores integer value
| char *formula     |  <-- Stores formula as string
| int is_formula    |  <-- Flag for formula existence
+--------------------+
\end{verbatim}

\subsection{Spreadsheet Grid (\texttt{spreadsheet\_t})}
\begin{verbatim}
+----------------------+
| spreadsheet_t       |
|----------------------|
| cell_t **cells      |  <-- 2D array of cells
| int rows, cols      |  <-- Grid dimensions
+----------------------+
\end{verbatim}

\subsection{Dependency Graph (DAG)}
\begin{verbatim}
    A1 --> B1 --> C1
           |
           v
          D1
\end{verbatim}
A Directed Acyclic Graph (DAG) is used to track dependencies, ensuring automatic updates when referenced values change.

\section{Function Flow}
\begin{verbatim}
User Input (CLI)
     |
     v
+----------------------+
| cl_interface.c       |  <-- Reads user input
|----------------------|
| print_spreadsheet()  |
| process_command()    |
+----------------------+
     |
     v
+----------------------+
| commands.c          |  <-- Executes commands
|----------------------|
| set_value()         |
| set_formula()       |
| compute_formula()   |
+----------------------+
     |
     v
+----------------------+
| input_handling.c    |  <-- Parses and validates input
|----------------------|
| parse_command()     |
| validate_formula()  |
+----------------------+
     |
     v
Spreadsheet Grid Updates
\end{verbatim}

\section{Cell Update Process}
\begin{verbatim}
1. User enters: "SET C1 = A1 + B1"
2. cl_interface.c processes command
3. commands.c updates value in spreadsheet_t
4. DAG updates dependencies
5. compute_formula() recalculates C1
6. Updated spreadsheet is displayed
\end{verbatim}


\section{Demo and Code Repository}
\begin{itemize}
    \item Demo Video: \url{https://csciitd-my.sharepoint.com/:v:/g/personal/ee1221699_iitd_ac_in/EU236_kxHIJKvjh3DnVTwHsB-FNCE-NoJKIi2N8mXT0qxw?e=9oDxFt&nav=eyJyZWZlcnJhbEluZm8iOnsicmVmZXJyYWxBcHAiOiJTdHJlYW1XZWJBcHAiLCJyZWZlcnJhbFZpZXciOiJTaGFyZURpYWxvZy1MaW5rIiwicmVmZXJyYWxBcHBQbGF0Zm9ybSI6IldlYiIsInJlZmVycmFsTW9kZSI6InZpZXcifX0%3D}
    \item GitHub Repository: \url{https://github.com/Parth150804/cop290-2402-C-rpm}
\end{itemize}

\section{Conclusion}
The project successfully implements a functional spreadsheet program with formula evaluation, error handling, and performance optimizations. Future improvements could include extending support for floating-point numbers and more complex functions.

\end{document}
